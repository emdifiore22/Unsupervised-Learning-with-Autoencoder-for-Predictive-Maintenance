\chapter{Conclusion}
As widely seen, anomaly detection is a data-driven approach for predictive maintenance. Since it gives the possibility to avoid downtime, to reduce costs related to unnecessary operations and to optimize maintenance procedures, in last years a lot of approaches have been developed and studied by researches. In this work, after a state-of-art on autoencoders solutions for anomaly detection task, a conditional autoencoder based framework has been presented to face unsupervised anomalous sound detection. Chapter 3 and Chapter 4 show, respectively, the details of the proposed framework components and experiments performed with it, demonstrating that the involvement of machine identifiers (conditioning) into the autoencoder learning process enables models training on sounds belonging to different machines of the same type (like different pumps), improving their detection capabilities. Moreover, the experimental part of the work shows that the conditioning can be done with a classical neural network, which is also compatible with the various types of encoder-decoder layers. To demonstrate this, experiments have been conducted on DCASE 2020 Task 2 Challenge dataset, which is already arranged to face the task in unsupervised way, replicating a real scenario in which normal sound clips represent most of the available data for training phase. Two instances of the proposed framework have been realized, the first based on convolutional autoencoder, while the second uses a recurrent approach, with LSTM layers in both encoder and decoder. Results have been evaluated using AUC and pAUC, as indicated by the challenge, and they highlight some improvements, with respect to the corresponding non-conditioned autoencoder versions, especially for pAUC.\\
Literature results and those collected during the experiments are really promising but improvements can still be made. In fact, regarding possible future works, in addition to a more complete hyperparameter optimization, different types of conditioning networks and operations could be attempted, together with the application of some pre-processing strategies to get better training performances, like noise reduction, with the aim of reducing the audio clips background noise surely present in factory environments, or audio data augmentation techniques, like pitching, time-shifting, etc. Obviously, this study could be also extended by the involvement of other, innovative, neural networks, like Variational Autoencoders (VAE) or Generative Adversarial Networks (GAN). Moreover, as the models are trained on audio clips recorded from different machines of the same type, in order to reach better performances, ensemble approaches should be analyzed, considering that there is not an hyperparameter set that is equally good for all of them.\\
In conclusion, an important consideration to do is that anomaly detection techniques based on audio clips are very innovative in predictive maintenance field. Therefore, is very important to validate various approaches, including the one proposed, in every day real industrial machines working scenarios, in order to study, in the most accurate and truthful way, their possible positive impact in maintenance procedures and costs.