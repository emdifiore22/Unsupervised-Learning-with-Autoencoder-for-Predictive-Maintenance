\chapter{Abstract} 

Predictive Maintenance is one of the most important field in the Industry 4.0 era. With the use of Internet of Things (IoT) tools and Information and Communication Technologies (ICT), a huge amount of data related to industrial machinery operating conditions can be collected. This enables and encourages the use of data-driven approaches for predictive maintenance, especially those based on machine learning and deep learning techniques. With this knowledge, maintenance activities can be planned in optimal way, reducing or even avoiding downtime and saving costs due to unnecessary maintenance procedures.\\
Anomaly detection is a set of techniques used to understand if conditions in which an industrial machine is working are anomalous or not, only using data collected with a various set of sensors. In last years, many researches started to use deep learning approaches for anomaly detection purposes, enveloped by the great results of these techniques in Computer Vision (CV) and Natural Language Processing (NLP). For example, one of the most used deep learning models is Autoencoder. In fact, even if it is mainly used for dimensionality reduction, autoencoder can be easily adapted to anomaly detection task especially because it allows to resolve its main problem: the absence of sufficient anomalous data to build a classifier. An autoencoder can be trained in unsupervised way, with the absence of anomalous data.\\ Usually, data collected using IoT sensors and then used for anomaly detection range from time-series temperature measurements to equipment vibrations, like bearings, collected by accelerometers. A recent trend is to train autoencoders for anomaly detection using sounds coming from machinery, collected by microphones. In literature, this task in called Anomalous Sound Detection (ASD). Therefore, in this text an unsupervised anomalous sound detection framework using autoencoders is presented.\\
To validate the framework, DCASE 2020 Task 2 Challenge dataset is used as a benchmark, because it natively designed for unsupervised anomalous sound detection task. In particular, two instances of the proposed framework are implemented: ID Conditioned Convolutional Autoencoder and ID Conditioned LSTM Autoencoder. Results are compared with the baseline model results, provided by the authors of the challenge, and with those found in documentations published by the participants.